\chapter{\LaTeX 文类介绍}
除了学习资料较少之外,
\LaTeX 并没有国际统一的编写规范,所以他人写的文类可能阅读性比较差;
而且编写和使用模板过程中的语言风格有着很大的区别,
即便在能使用现成模板熟练排版后,想要编写自己的模板也有一定的困难;
同时国内的宏包更新和普及比较慢,
这就导致部分模板中使用的宏包和命令在网上找不到对应的功能和使用方法。
这些都是在学习编写\LaTeX 模板时可能会遇到的问题。

一般来说,\LaTeX 模板在定义排版格式前会先声明以下内容:
版本信息、基本文类加载、模板选项声明、条件变量声明、宏包声明、配置文件的加载等。
后续格式的定义主要是命令和环境的定义和修改。
了解这些常识后,文类的学习会相对简单。

