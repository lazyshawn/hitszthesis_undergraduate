\chapter{绪论}

\section{课题背景及研究的目的和意义}
本项目根据哈工大(深圳)本科毕业论文撰写规范编写了\LaTeX 模板,
同时提供了完整的使用和修改手册,方便初学者使用和学习。

论文的排版对于学者来说虽然并不重要,但是对于学校和出版社等机构来说格式却很关键。
不同学生论文需要以同一排版要求上交给学校收集并管理,
出版社需要将不同作者的文章以同一格式排版并发布。
由于每个人都有不同的排版习惯,在没有同一要求的前提下会显得很随意,
这不便于对大量论文的管理。
即便有了统一的要求,在编写论文时可能也会出现少许格式错误,后期查错和修改十分繁琐;
而且如果官方修改了部分格式规范,
那么作者或者修改者需要对每一处涉及格式变化的部分进行修改。
很显然使用Word或类Word排版工具进行排版都会遇到这些麻烦,
而使用\LaTeX 排版可以实现内容和格式分离,从而很好地解决上述问题。
只要作者使用了官方提供的\LaTeX 模板,那么所有人的文章都是同一格式,
即便排版格式发生了变化,也只要更新官方的模板即可修改格式,
文章作者几乎不用考虑排版的问题,而机构的论文管理者也能轻易地排版所有论文。

国内外大部分高校都提供了各自的\LaTeX 论文模板,以供学生使用,
而我校并未有官方版本,只有部分热心的同学抽空编写的模板,可以暂时解决模板的问题。
另一方面,由于我校的优良传统,格式要求的变化是家常便饭,一日一更是小事(微笑脸),
那么这种由同学提供和维护模板很可能会突然断更,从而不再适用。

为了让更多同学学会使用\LaTeX 排版,同时又不用担心没有模板或模板过久等问题,
本文全面地介绍了一篇毕业论文各部分格式的排版方法,
希望能对\LaTeX 的普及贡献一份力量,也衷心祝愿我校有朝一日能为学生提供模板,
不再要学生为了愚蠢的格式问题毫无意义地修改论文。

\section{国内外研究现状}
\LaTeX 是一种理工科学术界常用的排版工具,因为其很好地实现了内容与格式分离,
所以学校和出版社等机构能很方便地管理其发表的论文格式,
作者只需要使用官方提供的模板填充论文内容,格式方面可以任由官方修改。
\LaTeX 出色的排版效果获得了国内外众多机构的青睐,尤其是国外的期刊出版社。

\section{研究现状分析}
目前国外普遍习惯使用\LaTeX 进行排版,而国内使用地较少。
由于国内使用者较少,所以相应的教学资料也零碎和稀缺,初学者起步艰难。

\section{本文的主要研究内容}
本课题的研究内容主要是从运动过程中版面设计、标题格式、正文格式、
浮动提环境、特殊章节等方面展开,同时结合实际案例介绍\LaTeX 模板的编写方法。

