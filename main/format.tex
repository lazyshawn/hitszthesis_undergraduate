%! TEX root = ../main.tex 
%! TEX program = xelatex

\chapter{排版格式的定义}

\section{版面设计}
版面就是书刊每一页的整面,由版心、页眉、页脚、边注以及四周空白组成,
他们被成为版面元素\cite[139]{man}。


\section{论文分区}
一篇论文的结构划分和书籍类似,一般分为书前、正文、附录和书后部分。
这些结构的主要区别在于章节和页码的编号方式。
\begin{itemize}
  \item frontmatter:\ 关闭章节序号,页码使用罗马数字;
  \item mainmatter:\ 开启章节序号计数,重置页码,页码使用阿拉伯数字;
  \item appendix:\ 重置章节序号计数, 章节序号使用字母,对页码没有影响;
  \item backmatter:\ 关闭章节序号,对页码没有影响。
\end{itemize}


\section{字体声明}
在Linux下使用\LaTeX 时要先确保已经安装了需要的字体,
因为Linux一般没有自带常用的中英文字体。

使用下面的命令可以定义新的中文字体,
添加中文字体需要先加载系统中的字体,再通过
\begin{lstlisting}[style=Latex]
  \setCJKfamilyfont{#1}[Opt]{#2}
  \newcommand*{\kai}{\CJKfamily{kai}}
\end{lstlisting}
其中第一条命令将系统中的\#2字体(如simkai.ttf)定义为中文字体\#1;
使用$\backslash$setCJKfamilyfont命令可以修改文章字体,
第二条命令只是简化这一字体命令。


\section{封面排版}
封面的排版比较复杂,如果觉得麻烦可以直接在Word中排版好,
最后导出PDF文件并与\LaTeX 导出的PDF合成即可。
当然,直接用\LaTeX 也是可行的。

在排版封面时主要用到的几个命令是
$\backslash$parbox、$\backslash$makebox、$\backslash$uline;
主要用到的环境有:
tabular、center、minipage以及浮动体环境。
组合使用这些命令和环境可以分别排版封面中的各个子区域,
最后使用垂直间距命令$\backslash$vspace调整各子区域之间的相对位置。


\section{页眉页脚}
页眉页脚的设置需要用到fancy宏包。
下面是定义正文部分页眉页脚格式的代码。
\begin{lstlisting}[style=latex]
  \fancypagestyle{fancy@main}{\fancyhf{}%
    \fancyfoot[C]{\zihao{-5} -~\thepage~-}
    \fancyhead[C]{\yemei \hitsz@info@school \hitsz@token@shenzhen \hitsz@info@thesis}
    \renewcommand{\headrule}{%
    \hrule width\headwidth height2.276pt \vspace{1pt}\hrule width\headwidth}
    \setlength{\headheight}{15pt}
  }
\end{lstlisting}
其中,第一行定义了名为fancy@main的页眉格式;
第二行和第三行分别设置了页脚、页眉的内容和字体;
四至六行用于生成双页眉线(亦称文武线)。


\section{章节标题}
使用$\backslash$setcounter\{secnumdepth\}\{3\}命令可以设置正文中标题的层次深度。
BOOK文类中part深度为0,chapter深度为1,section等标题的深度以此类推。
这一设置可以用来取消标题编号,当secnumdepth被设为-1时,
所有标题将不会编号(目录中也不会编号),
注意这只是不显示编号,并不会影响后续章节的序号。
在需要编号的章节中重新赋值secnumdepth即可恢复编号。

使用$\backslash$ctexset命令可以设置各级标题的格式,
其赋值方式与json文件类似,可定义内容参见参考文献\cite[193]{man}。

\begin{lstlisting}[style=latex]
\ctexset{%
  chapter={
    name={第,章},
    afterindent=true,
    % 文类book将每一新章另起一页,该页的默认版式为plain
    pagestyle={fancy@main},
    beforeskip={18.74658bp},
    afterskip={24.74658bp},
    format={\centering\heiti\zihao{-2}},
    nameformat=\bfseries,
    titleformat=\bfseries,
  },
}
\end{lstlisting}


\section{目录格式修改}
目录格式修改可以直接使用宏包,也可以重定义$\backslash$tableofcontents命令。
book文类中对章节目录命令的定义如下:

\begin{lstlisting}[style=latex]
% --- 以章节格式添加目录
\renewcommand\tableofcontents{%
  \if@openright\cleardoublepage\else\clearpage\fi
  \phantomsection
  \chapter*{\contentsname}
  \normalsize\@starttoc{toc}
}

% --- 目录条目格式的设置
\def\@pnumwidth{1.55em}  % 页码宽度
\def\@tocrmarg{2.55em}   % 标题行左缩进宽度
\def\@dotsep{0.3}
% 目录中章节的层次深度
\setcounter{tocdepth}{2}
\end{lstlisting}
首先通过重定义$\backslash$tableofcontents命令将目录以章节的格式加入目录条目;
三条长度命令分别
定义预留页码宽度即为排印页码而预留的空白宽度$\backslash$@pnumwidth为1.55em;
定义左缩进宽度$\backslash$@tocrmarg为2.55em;
定义指引线中相邻两圆点之间的距离$\backslash$@dotsep为4.5,默认单位为mu。

\begin{lstlisting}[style=latex]
% 目录中的章标题格式
\renewcommand*\l@chapter[2]{%
  \ifnum \c@tocdepth >\m@ne
    \addpenalty{-\@highpenalty}%
    % \vskip 1em \@plus\p@
    \setlength\@tempdima{4em} % 序号盒子宽度 (第n章)
\end{lstlisting}
如果设定的目录层度大于$\backslash$m@ne(-1),
则可以在章节条目前换页,但并不鼓励这么做。
在本级条目前添加约2.25em的垂直空白,以与前一条目分隔。
设置序号盒子的宽度,如"第n章"所占的宽度。

\begin{lstlisting}[style=latex]
    \begingroup
      \parindent \z@ \rightskip \@pnumwidth
      \parfillskip -\@pnumwidth
      \leavevmode %\bfseries
      \advance\leftskip\@tempdima
      \hskip -\leftskip
\end{lstlisting}
创建一个组合,段落首行缩进为0,段落右侧向左缩进$\backslash$@pnumwidth,
段落最后一行文本向右凸出,使页码凸出于条目段落右侧。
$\backslash$leavevmode 从垂直模式切换到水平模式,即开始一个段落,
用于排列章标题,后续可设置标题字体格式。
随后将标题段落整体从左向右缩进$\backslash$tempdima,
再将段落首行反向缩进相等距离,以使得章节标题首行与版心左侧对齐,
使除章节序号外的标题段落的各行文本左对齐。

\begin{lstlisting}[style=latex]
      \heiti #1 \nobreak\hfil % 章节标题内禁止换行
      \leaders\hbox{$\m@th\mkern \@dotsep mu\hbox{.}\mkern \@dotsep mu$}\hfill
      \nobreak\hb@xt@\@pnumwidth{\hss #2}\par
      \penalty\@highpenalty
    \endgroup
  \fi}
\end{lstlisting}
\#1参数代表章标题内容,\#2代表章标题页的页码。
先设置章标题不允许换行,字体设为黑体。
添加一个水平盒子,以$\backslash$@dotsep为间隔排满指引符。
设置标题末尾与页码之间禁止换行,随后将页码放入水平盒子并使其右对齐。
尽量避免在章节条目之后换页,结束组合。

\begin{lstlisting}[style=latex]
% 子标题的格式设置
\renewcommand*\l@section{\@dottedtocline{1}{1em}{1.8em}}
\renewcommand*\l@subsection{\@dottedtocline{2}{2em}{2.5em}}
\renewcommand*\l@subsubsection{\@dottedtocline{3}{3\ccwd}{3.1em}}
\end{lstlisting}
系统还提供了条目行命令,使用格式为
$\backslash$@dettedtocline\{层次深度\}\{缩进宽度\}\{序号宽度\},
常用于定义格式较简单的节及节以下的条目格式。


\section{参考文献格式}
\subsection{文献列表格式}

\subsection{文献排版及引用}


\section{其他特殊章节}


\section{公式排版}


\section{图表格式修改}


\section{代码格式的定义}


\section{其他宏包的配置}


