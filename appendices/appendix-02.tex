\appchapter{研究现状}
在手操作中,即重新定位机器人手中的物体,
一直是机器人学中的一个长期研究课题,涉及到建模、运动规划和控制等多个方面。
Tournassoud 等人的早期研究展示了可以通过在一个表面上通过反复地抓取
和放置来实现再抓取 $^{[6]}$。然而,这个过程可能会很耗时,
并且可行的再抓取姿态由物体能稳定地放置在该表面上的姿态决定。
因此,研究人员很快意识到需要更高级的灵巧操作技巧。
一种提出的解决方案是控制手指的运动,
通过滚动、滑动和在手指间移动来实现被抓取物体的重新定位$^{[7, 8]}$。

另一方面,其他研究提出了利用环境协助在手操作的想法。Brock 等人是最早
提出在外力作用下通过物体的受控滑动来增强机器人灵巧性的人之一$^{[9]}$。 Dafle 等
人定义了外部灵巧性这一概念,并展示了一个低自由度机器人手仍然可以利用重
力和环境约束等外部资源完成一组不同的原始动作,进而实现复杂的在手操作$^{[1]}$。
这种利用机器人环境的想法也被用于抓取情景中$^{[10]}$。

这些研究强调了接触和摩擦建模作为灵巧操作的核心内容的重要性。 
Goyal 提出了极限曲面的概念,极限曲面描述了可以让被抓取物体发生滑动的力旋的边界,
以及滑动发生时物体的运动方向的边界 $^{[11]}$。Howe 等人进一步发展了这些思想,
并提出了在操纵规划和控制时估计极限面的便于计算的方法$^{[12]}$。

在机械系统的摩擦识别和补偿方案设计中,也对摩擦建模进行了研究$^{[13, 14]}$。
然而,在这些研究中发展的控制技术不能直接应用到我们的工作中 ,
因 为 他 们 将 摩 擦 视 为 附加扰动,而在我们的研究中,摩擦代表一个控制输入。
从这个角度看我们的控制器与汽车上的防抱死制动系统(ABS)有一定的相似之处,
然而这些工作的目的是最大限度地提高轮胎和路面之间的牵引力$^{[15]}$。

触觉感知在机器人操作中也发挥了重要作用,部分原因是它在人类执行的最
基本的拾取和放置操作任务中发挥了重要作用$^{[16]}$。
许多研究都是通过触觉感知来解决滑动检测问题,
然而,这些研究的主要目的是在抓握控制中防止滑动,而不是控制滑动$^{[17, 18]}$。
一些工作也提出了摩擦参数的在线估计方法$^{[18]}$,但重点放在滑动摩擦力上,
而不是像我们的例子中的摩擦力矩。

近期有学者对具有外部灵巧性的在手操作进行了建模、机械设计和运动规划
等方面的研究。Dafle 等人研究了抓握推送的力学,分析了推送器与被控物体接触
的几何形状对物体滑动的影响 $^{[2]}$。Dafle 等人还设计了特别的指尖,使机器人能
够轻松地转换抓取方式,实现稳定地抓取物体或者让物体自由转动$^{[19]}$。

Shi 等人提出了一种运动规划框架,
该框架可以确定让物体相对于机器人手滑动一定距离时机器人手所需的加速度 $^{[4]}$。
这项工作解决了一个类似于我们的场景,即一个物体被两个手指捏住,
还能在三个平动自由度上调整物体相对于机器人手的位置。
虽然这项研究中的仿真实验验证了其提出的方法,但实验并没有达到预期的效果,
部分原因是缺乏反馈控制和触觉感知。

与我们的工作密切相关的还有 Holladay 等人提出的开环旋转框架$^{[3]}$。在他的
研究中,机器人首先在夹持在一个物体的表面,然后按照预先计算的运动轨迹将其
举起。如果使用开环控制,没有对物体姿态的在线跟踪,也没有对夹持力的控制,
那么物体只能在一组离散的稳定姿态之间旋转。

与近年来采用开环运动规划策略的文献$^{[3,4]}$相比,我们着重于采用具有在
线视觉跟踪和触觉传感的自适应反馈控制来控制抓取力。

